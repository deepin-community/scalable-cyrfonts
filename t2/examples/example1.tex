% This is the file example1.tex of the T2 package.

\documentclass{article}

% The following command overrides the default T2A encoding.  If you
% don't have new LH fonts with X2 and T2* encodings, you have to use
% this command. It will select old lh* fonts with the LCY encoding.
% You may also use OT2 7-bit cyrillic encoding.
%
% The general rule: always call the `fontenc' package *before* the BABEL
% package, and also always use encoding which contains latin letters as
% a last option to the `fontenc' package (i.e., X2 or OT2 must not
% appear as a last option!).
%
% Uncomment the following line if you don't have the new LH fonts.
%
%\usepackage[LCY,OT1]{fontenc}

% You may specify the (default) input encoding for the Russian letters.
% Some of available encodings are (see cyinpenc.dtx for a full list):
%   cp866   --- MS-DOS Russian codepage
%     cp866av   Alternative Variant of cp866
%     cp866mav  Modified Alternative Variant of cp866
%     cp866nav  New Alternative Variant of cp866
%   cp1251  --- MS-Windows Cyrillic codepage
%   cp855   --- MS-DOS Cyrillic codepage
%   koi8-r  --- koi8-r Russian codepage (as of RFC1489)
%     isoir111
%     koi8-ru
%     koi8-u    koi8-u Ukrainian codepage
%   8859-5  --- ISO 8859-5 Cyrillic codepage
%   maccyr  --- Apple Macintosh Cyrillic codepage (AKA MS cp10007)
%   macukr  --- Apple Macintosh Ukrainian codepage

\usepackage[cp1251,koi8-r]{inputenc}

\usepackage[english,russian]{babel}   % load Babel setup for English
                                      % and Russian languages;
                                      % the latter is the default.

\begin{document}

Here goes some text in koi8-r input encoding.
It will look as a `garbage' of cyrillic glyphs if the X2 encoding is used,
because X2 does not contain latin letters. :-)
However, it will be Ok with the T2A encoding which appeared in lhfnt v3.19.

% To switch the current language (and font encoding) to English,
% use \English macro (or it's synonym: \Eng)
% To switch the current language (and font encoding) to Russian,
% use \Russian macro (or it's synonym: \Rus)
% Switching languages will, in particular, set the correct hyphenation.

% IMPORTANT:
% If you are using X2 encoding (which does not contain latin letters), you
% have to use language-switching commands (which also select the correct fonts).
% You also have to use language-switching commands to have right hyphenation
% anyway (unless you're using a combined `russian-english' language).

% You may use several different input encodings in one document! :-)

\inputencoding{cp1251}

Here goes some text in cp1251 input encoding

% You may also use several different Cyrillic font encodings in one document!

\fontencoding{T2A}\selectfont

Proba Pera

\end{document}
